\documentclass[t,aspectratio=169]{beamer}
\usetheme{metropolis}

%%% presentation below

\title{Ansible}
\subtitle{Configuration Management System done right}
\author{Fabio Alessandro Locati}
\date{25 October 2016}

\begin{document}

\maketitle

\begin{frame}
    \frametitle{Outline}
    \tableofcontents
\end{frame}

\section{Intro}
\begin{frame}
    \frametitle{About me}
    \begin{itemize}
        \item<2-> IT Consultant since 2004
        \item<3-> Ansible user since 2013
    \end{itemize}
\end{frame}

\begin{frame}
    \frametitle{Today's problems}
    \begin{itemize}
        \item<2-> Auditability
        \item<3-> Job-hopping
        \item<4-> Speed
        \item<5-> Scalability
        \item<6-> Horizontal scaling (IaaS "cloud")
        \item<7-> Expected QoS
    \end{itemize}
\end{frame}

\section{Automation}

\begin{frame}
    \frametitle{Advantages}
    \begin{itemize}
        \item<2-> Infrastructure as Code
        \begin{itemize}
            \item<3-> Code is the infrastructure documentation*
            \item<4-> Simplify auditability
        \end{itemize}
        \item<5-> Infrastructures with no humans with \textit{root} powers
        \item<6-> Easy and quick to scale out
    \end{itemize}
\end{frame}

\section{Automation Options}
\begin{frame}
    \frametitle{Agent (pull) vs agent-less (push)}
    \begin{block}<2->{Agent}
    An Agent is a daemon that runs on every controlled machine and that will check with the server (master) every N minutes to ensure that the host is aligned with the latest configuration version. If this is not the case, the Agent will download the lastest configuration version and apply it.
    \end{block}
    \begin{itemize}
        \item<3-> Advantages
        \begin{itemize}
            \item<4-> High performance during commands execution
            \item<5-> Connection between clients and server is client managed
        \end{itemize}
        \item<6-> Disadvantages
        \begin{itemize}
            \item<7-> Forces the master to be in the least secure network segment
            \item<8-> Resources are used even if no changes are being applied
            \item<9-> More daemons to take care of
            \item<10-> Chicken and Egg problem 
        \end{itemize}
    \end{itemize}
\end{frame}

\begin{frame}
    \frametitle{Idempotence}
    \begin{definition}{Idempotence}
    is the property of certain operations in mathematics and computer science, that can be applied multiple times without changing the result beyond the initial application. 
    \end{definition}
\end{frame}

\subsection{Chef}
\begin{frame}
    \frametitle{Chef}
    \begin{itemize}
        \item<2-> Written in Ruby 
        \item<3-> Pull mode only
        \item<4-> Advantages
        \begin{itemize}
            \item<5-> Very well integrated with git
            \item<6-> Rich collection of available modules
            \item<7-> Easy to install
        \end{itemize}
        \item<8-> Disadvantages
        \begin{itemize}
            \item<9-> Code driven
            \item<10-> Complex tool
            \item<11-> Steep learning curve
        \end{itemize}
    \end{itemize}
\end{frame}

\subsection{Puppet}
\begin{frame}
    \frametitle{Puppet}
    \begin{itemize}
        \item<2-> Written in Ruby
        \item<3-> Mainly pull mode
        \item<4-> Advantages
        \begin{itemize}
            \item<5-> Very large user base
            \item<6-> High number of modules available
        \end{itemize}
        \item<7-> Disadvantages
        \begin{itemize}
            \item<8-> Steep learning curve
            \item<9-> Complex to setup
        \end{itemize}
    \end{itemize}
\end{frame}

\subsection{SaltStack}
\begin{frame}
    \frametitle{SaltStack}
    \begin{itemize}
        \item<2-> Written in Python
        \item<3-> Both push and pull mode
        \item<4-> Advantages
        \begin{itemize}
            \item<5-> Very consistent use of YAML for input, output, and configs
            \item<6-> Strong community
            \item<7-> Highly scalable and resilient
            \item<8-> Very good introspection tools
        \end{itemize}
        \item<9-> Disadvantages
        \begin{itemize}
            \item<10-> Very complex to setup
            \item<11-> Very steep learning curve
        \end{itemize}
    \end{itemize}
\end{frame}

\subsection{Ansible}
\begin{frame}
    \frametitle{Ansible}
    \begin{itemize}
        \item<2-> Written in Python
        \item<3-> Mainly push mode
        \item<4-> Advantages
        \begin{itemize}
            \item<5-> Infrastructure as \textbf{Data} (in YAML format)
            \item<6-> Very gentle learning curve
            \item<7-> Very simple setup
            \item<8-> Balanced tool
        \end{itemize}
        \item<9-> Disadvantages
        \begin{itemize}
            \item<10-> Not very good introspection tools
            \item<11-> Community is young
        \end{itemize}
    \end{itemize}
\end{frame}

\section{Ansible}
\begin{frame}
    \frametitle{Ansible terminology}
    \begin{itemize}
        \item<2->\textbf{Host}: Target of the execution
        \item<3->\textbf{Module}: Modules can control system resources, like services, packages, or files (anything really), or handle executing system commands.
        \item<4->\textbf{Module library}: Default set of modules coming with Ansible basic installation
        \item<5->\textbf{Task}: An istance of a Module 
        \item<6->\textbf{Role}: A way to abstract a collection of tasks that has a specific role and is idempotent
        \item<7->\textbf{Playbook}: A collection of Tasks and Roles that could be idempotent (or not)
    \end{itemize}
\end{frame}

\begin{frame}[fragile]
    \frametitle{Ansible infrastructure}
    \begin{semiverbatim}
+---------------------+
|     GIT Server      |
+---------------------+     .
           |                .
           |                .   +---------------------+
           V                |-->|   Controlled Host   |
+---------------------+     |   +---------------------+
| Ansible Controller  |-----|
+---------------------+     |   +---------------------+
                            |-->|   Controlled Host   |
                            .   +---------------------+
                            .
                            .
    \end{semiverbatim}
\end{frame}

\begin{frame}
    \frametitle{Infrastructure as Data}
    \begin{itemize}
        \item<2-> Really simple to write
        \item<3-> Even simpler to read
        \item<4-> Only the bit important to you need to be written
    \end{itemize}
\end{frame}

\begin{frame}[fragile]
    \frametitle{Example of syntax}
    \begin{semiverbatim}
\textcolor{green}{---}
\textcolor{green}{- hosts:} \textcolor{blue}{all}
\textcolor{green}{  become:} \textcolor{blue}{True}
\textcolor{green}{  tasks:}
\textcolor{green}{  - name:} \textcolor{blue}{Ensure mysql is installed}
\textcolor{green}{    yum:}
\textcolor{green}{      name:} \textcolor{blue}{mysql}
\textcolor{green}{      state:} \textcolor{blue}{present}
\textcolor{green}{  - name:} \textcolor{blue}{Ensure tom user is present}
\textcolor{green}{    user:}
\textcolor{green}{      name:} \textcolor{blue}{tom}
\textcolor{green}{      state:} \textcolor{blue}{present}
    \end{semiverbatim}
\end{frame}

\begin{frame}
    \frametitle{Usual deployment process}
    \begin{itemize}
        \item<2-> Automate few actions with Ansible Playbooks
        \item<3-> Create Ansible Roles for the setup of a simple machine type
        \item<4-> Rollout of the first machines completely managed with Ansible
        \item<5-> Migration of all machines to Ansible
    \end{itemize}
\end{frame}

\begin{frame}
    \frametitle{Additional resources}
    \begin{itemize}
        \item Laboratorio ICT, 14:00 - Workshop su come automatizzare l'IT con Ansible
        \item Slides: https://slides.fale.io/20161025-en-ansible.pdf
        \item Official documentation: http://docs.ansible.com
        \item Videos: https://www.ansible.com/videos
        \item Whitepapers: https://www.ansible.com/whitepapers
        \item Ebooks: https://www.ansible.com/ebooks
    \end{itemize}
\end{frame}

%\makethanks

\end{document}
